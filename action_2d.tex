\documentclass{article}
\usepackage{latexsym}
\usepackage{geometry}
\usepackage{amsmath}
\geometry{letterpaper}

%% common definitions
\newcommand{\p}{\partial}
\newcommand{\wt}{\widetilde}
\newcommand{\ov}{\overline}
\newcommand{\mc}[1]{\mathcal{#1}}
\newcommand{\md}{\mathcal{D}}


\newcommand{\GeV}{{\rm GeV}}
\newcommand{\eV}{{\rm eV}}
\newcommand{\Heff}{{\mathcal{H}_{\rm eff}}}
\newcommand{\Leff}{{\mathcal{L}_{\rm eff}}}
\newcommand{\el}{{\rm EM}}
\newcommand{\uflavor}{\mathbf{1}_{\rm flavor}}
\newcommand{\lgr}{\left\lgroup}
\newcommand{\rgr}{\right\rgroup}

\newcommand{\LUV}{\Lambda_{\rm UV}}
\newcommand{\LNC}{\Lambda_{\rm NC}}
\newcommand{\LQCD}{\Lambda_{\rm QCD}}
\newcommand{\Mpl}{M_{\rm Pl}}
\newcommand{\suc}{{{\rm SU}_{\rm C}(3)}}
\newcommand{\sul}{{{\rm SU}_{\rm L}(2)}}
\newcommand{\sutw}{{\rm SU}(2)}
\newcommand{\suth}{{\rm SU}(3)}
\newcommand{\ue}{{\rm U}(1)}
%%%%%%%%%%%%%%%%%%%%%%%%%%%%%%%%%%%%%%%
%  Slash character...
\def\slashed#1{\setbox0=\hbox{$#1$}             % set a box for #1
   \dimen0=\wd0                                 % and get its size
   \setbox1=\hbox{/} \dimen1=\wd1               % get size of /
   \ifdim\dimen0>\dimen1                        % #1 is bigger
      \rlap{\hbox to \dimen0{\hfil/\hfil}}      % so center / in box
      #1                                        % and print #1
   \else                                        % / is bigger
      \rlap{\hbox to \dimen1{\hfil$#1$\hfil}}   % so center #1
      /                                         % and print /
   \fi}                                        %

%%EXAMPLE:  $\slashed{E}$ or $\slashed{E}_{t}$

%%

\begin{document}

\section{$\mathcal{N}=2$ Supersymmetry}

 {\it Superorientational} zero-modes of the NA string in ${\mathcal N}=2$ U(N) SQCD.
\begin{align}
\label{N2_sorient}
%
\notag
\overline{\psi}_{\dot{2}Ak} & ~~=~~ \frac{\phi_1^2 ~-~ \phi_2^2}{\phi_2} \cdot n \overline{\xi}_L   \\
%
\notag
\overline{\wt{\psi}}_{\dot{1}}^{kA}  & ~~=~~ - \frac{\phi_1^2 ~-~ \phi_2^2}{\phi_2} \cdot \xi_R n^*  \\
%
\lambda^{11\ SU(N)} & ~~=~~ i \sqrt{2}\, \frac{ x^1 ~-~ i\, x^2 }{r^2} \frac{\phi_1}{\phi_2} f_N \cdot n \overline{\xi}_L \\
%
\notag
\lambda^{22\ SU(N)} & ~~=~~ - i \sqrt{2}\, \frac{ x^1 ~+~ i\, x^2 }{r^2} \frac{\phi_1}{\phi_2} f_N \cdot \xi_R n^* 
\end{align}

 {\it Supertranslational} zero-modes of the NA string in ${\mathcal N}=2$ U(N) SQCD.
\begin{align}
\label{N2_strans}
%
\notag
\ov{\psi}_{\dot{2}}	& ~~=~~  -\,  2\sqrt{2}\, \frac{x_1 ~+~ i x_2}{N r^2} \,
		\lgr \frac{1}{N} \phi_1 ( f + (N-1) f_N ) ~+~ \frac{N-1}{N} \phi_2 ( f - f_N ) ~+~ \right.
		\\
%
\notag
			& \phantom{~~=~~  -\,  2\sqrt{2}\, \frac{x_1 ~+~ i x_2}{N r^2} \,\lgr \right.}
			\left( nn^* ~-~ 1/N \right )
			\Bigl\{ \phi_1 ( f + ( N-1 ) f_N ) ~-~ \phi_2 ( f - f_N) \Bigr\}
		\left. \rgr\, \zeta_L 
		\\
%
\notag
\ov{\wt{\psi}}_{\dot{1}} & ~~=~~    2\sqrt{2} \, \frac{x_1 ~-~ i x_2}{N r^2} \,
		\lgr \frac{1}{N} \phi_1 ( f + (N-1) f_N ) ~+~ \frac{N-1}{N} \phi_2 ( f - f_N ) ~+~ \right.
		\\
%
\notag
			& \phantom{~~=~~    2\sqrt{2} \, \frac{x_1 ~-~ i x_2}{N r^2} \,\lgr \right.}
			\left( nn^* ~-~ 1/N \right )
			\Bigl\{ \phi_1 ( f + ( N-1 ) f_N ) ~-~ \phi_2 ( f - f_N) \Bigr\}
		\left. \rgr\, \zeta_R
		\\
%
\lambda^{11\ U(1)} 	& ~~=~~ -\, \frac{i g_1^2}{2} \lgr (N-1)\phi_2^2  ~+~ \phi_1^2 ~-~ N\xi \rgr \, \zeta_L 
		\\
%
\notag
\lambda^{22\ U(1)} 	& ~~=~~ +\, \frac{i g_1^2}{2} \lgr (N-1)\phi_2^2  ~+~ \phi_1^2 ~-~ N\xi \rgr \, \zeta_R 
		\\
%
\notag
\lambda^{11\ SU(N)}	& ~~=~~ -\, {i g_2^2}\, ( nn^* ~-~ 1/N )\, \lgr \phi_1^2 ~-~ \phi_2^2 \rgr\, \zeta_L
		\\
%
\notag
\lambda^{22\ SU(N)}	& ~~=~~ +\, {i g_2^2}\, ( nn^* ~-~ 1/N )\, \lgr \phi_1^2 ~-~ \phi_2^2 \rgr\, \zeta_R
\end{align}

	U(1) gauge fields are normalized as follows:
\begin{equation}
\label{abel_norm}
	A^{U(1)}_\mu ~~=~~ \frac 1 2 A^{\rm (abel)}_\mu ~, 
	\qquad\qquad   \lambda^{f\alpha\ U(1)} ~~=~~ \frac 1 2 \lambda^{f\alpha}_{\rm (abel)}
\end{equation}


\pagebreak

 ``Conjugate'' superorientational zero-modes of the NA string in ${\mathcal N}=2$ U(N) SQCD.
\begin{align}
\label{n2_sorient_conj}
%
\notag
{\psi}_{2} & ~~=~~ \frac{\phi_1^2 ~-~ \phi_2^2}{\phi_2} \cdot \xi_L n^*   \\
%
\notag
\wt{\psi}_{1}  & ~~=~~ - \frac{\phi_1^2 ~-~ \phi_2^2}{\phi_2} \cdot n \ov{\xi}_R  \\
%
\ov{\lambda}^{\dot{1}\ SU(N)}_{\ 1} & ~~=~~ - i \sqrt{2}\, \frac{ x^1 ~+~ i\, x^2 }{r^2} \frac{\phi_1}{\phi_2} f_N \cdot \xi_L n^* \\
%
\notag
\ov{\lambda}^{\dot{2}\ SU(N)}_{\ 2} & ~~=~~  i \sqrt{2}\, \frac{ x^1 ~-~ i\, x^2 }{r^2} \frac{\phi_1}{\phi_2} f_N \cdot n \ov{\xi}_R 
\end{align}

 ``Conjugate'' {\it supertranslational} zero-modes of the NA string in ${\mathcal N}=2$ U(N) SQCD.
\begin{align*}
%
\psi_{2}	& ~~=~~  -\,  2\sqrt{2}\, \frac{x_1 ~-~ i x_2}{N r^2} \,
		\lgr \frac{1}{N} \phi_1 ( f + (N-1) f_N ) ~+~ \frac{N-1}{N} \phi_2 ( f - f_N ) ~+~ \right.
		\\
%
			& \phantom{~~=~~  -\,  2\sqrt{2}\, \frac{x_1 ~-~ i x_2}{N r^2} \,\lgr \right.}
			\left( nn^* ~-~ 1/N \right )
			\Bigl\{ \phi_1 ( f + ( N-1 ) f_N ) ~-~ \phi_2 ( f - f_N) \Bigr\}
		\left. \rgr\, \ov{\zeta}_L
		\\
%
\wt{\psi}_{1} & ~~=~~    2\sqrt{2} \, \frac{x_1 ~+~ i x_2}{N r^2} \,
		\lgr \frac{1}{N} \phi_1 ( f + (N-1) f_N ) ~+~ \frac{N-1}{N} \phi_2 ( f - f_N ) ~+~ \right.
		\\
%
			& \phantom{~~=~~    2\sqrt{2} \, \frac{x_1 ~+~ i x_2}{N r^2} \,\lgr \right.}
			\left( nn^* ~-~ 1/N \right )
			\Bigl\{ \phi_1 ( f + ( N-1 ) f_N ) ~-~ \phi_2 ( f - f_N) \Bigr\}
		\left. \rgr\, \ov{\zeta}_R
		\\
%
\ov{\lambda}^{\dot{1}\ U(1)}_{\ 1} 	& ~~=~~ +\, \frac{i g_1^2}{2} \lgr (N-1)\phi_2^2  ~+~ \phi_1^2 ~-~ N\xi \rgr \, \ov{\zeta}_L 
		\\
%
\ov{\lambda}^{\dot{2}\ U(1)}_{\ 2} 	& ~~=~~ -\, \frac{i g_1^2}{2} \lgr (N-1)\phi_2^2  ~+~ \phi_1^2 ~-~ N\xi \rgr \, \ov{\zeta}_R 
		\\
%
\ov{\lambda}^{\dot{1}\ SU(N)}_{\ 1}	& ~~=~~ +\, {i g_2^2}\, ( nn^* ~-~ 1/N )\, \lgr \phi_1^2 ~-~ \phi_2^2 \rgr\, \ov{\zeta}_L
		\\
%
\ov{\lambda}^{\dot{2}\ SU(N)}_{\ 2}	& ~~=~~ -\, {i g_2^2}\, ( nn^* ~-~ 1/N )\, \lgr \phi_1^2 ~-~ \phi_2^2 \rgr\, \ov{\zeta}_R
\end{align*}

\pagebreak
	The action as obtained from substitution of the {\it superorientational} zero-modes and
	{\it supertranslational} zero-modes:

\begin{align*}
%
\mc{L}_{\rm superorient} ~~=~~ &
	\frac{4\pi}{g_2^2}\,
	\int dx^0 dx^3 
	\left\lgroup 
	\ov{\xi}_L \, i (\p_0 ~+~ i \p_3 ) \xi_L ~~+~~ \ov{\xi}_R \, i (\p_0 ~-~ i \p_3 ) \xi_R 
	\right\rgroup 
	\\
%
\mc{L}_{\rm supertrans} ~~=~~ &
	(?) \pi \xi\,
	\int dx^0 dx^3
	\lgr
		\ov{\zeta}_L \, i (\p_0 ~+~ i \p_3 ) \zeta_L ~~+~~ 
		\ov{\zeta}_R \, i (\p_0 ~-~ i \p_3 ) \zeta_R 
	\rgr
\end{align*}

The normalization is chosen as follows:
\[
	n^{*}_l \, n^l ~~=~~ 1 \qquad\qquad\qquad\qquad \chi_{R,L} ~~=~~ \xi_{R,L} n^* ~~+~~ n \ov{\xi}_{R,L}
\]


\pagebreak

\section{$\mathcal{N}=1$ Supersymmetry}

\subsection{Supertranslational zero-modes}
\newcommand{\loU}{\lambda_0^{\rm U(1)}}
\newcommand{\llU}{\lambda_1^{\rm U(1)}}
\newcommand{\loN}{\lambda_0^{\rm SU(N)}}
\newcommand{\llN}{\lambda_1^{\rm SU(N)}}
\newcommand{\poU}{\psi_0^{\rm U(1)}}
\newcommand{\plU}{\psi_1^{\rm U(1)}}
\newcommand{\poN}{\psi_0^{\rm SU(N)}}
\newcommand{\plN}{\psi_1^{\rm SU(N)}}

We define the profiles $ \loU $, $ \llU $, $ \loN $, $ \llN $, $ \poU $, $ \plU $, $ \poN $, and $ \plN $ 
as the generalization of the similar expressions for the U(2) theory:
\begin{align*}
%
	\lambda^{22\ \rm U(1)} & ~~=~~ \loU\, \zeta_R ~+~ \llU\, \frac{x^1 + i x^2}{r} \ov{\zeta}{}_R 
	\\
%
	\lambda^{22\ \rm SU(N)} & ~~=~~ \lgr  \loN\, \zeta_R ~+~ \llN\, \frac{x^1 + i x^2}{r} \ov{\zeta}{}_R \rgr
					( nn^* ~-~ 1/N )
	\\
%
	\ov{\wt{\psi}}{}_{\dot{1}} & ~~=~~ \frac{1}{2} \frac{x^1 - i x^2}{r}
				\lgr  \poU ~+~ N (nn^* ~-~ 1/N) \poN \rgr \zeta_R \\
				   & 
				~~+~~ \frac{1}{2} \lgr  \plU  ~+~ N (nn^* ~-~ 1/N) \plN \rgr  \ov{\zeta}{}_R
\end{align*}

The equations for the profiles are (the signs in front of $\mu$-terms are positive):
\begin{align}
%
\notag
&
	-\, \p_r \loU ~+~ \frac{i g_1^2}{4\sqrt{2}} 
			\lgr \poU (\phi_1 + \phi_2) ~+~ (N-1) \poN (\phi_1 - \phi_2) \rgr 
				~+~ g_1^2 \mu \llU    ~~=~~ 0
	\\
%
\notag
&
	-\, \p_r \llU ~-~ \frac{1}{r}\llU 
	~+~ \frac{i g_1^2}{4\sqrt{2}} 
	    \lgr \plU (\phi_1 + \phi_2) ~+~ (N-1) \plN (\phi_1 - \phi_2) \rgr 
	~+~ g_1^2 \mu \loU ~~=~~ 0
	\\
%
\notag
&
	-\, \p_r \loN ~+~ 
	\frac{i g_2^2}{2\sqrt{2}}
		\lgr \poU (\phi_1 - \phi_2) ~+~ \poN ( (N-1) \phi_1 + \phi_2 ) \rgr 
	~+~ g_2^2 \mu \llN ~~=~~ 0
	\\
%
\notag
&
	-\, \p_r \llN ~-~ \frac{1}{r}\llN
	~+~ \frac{i g_2^2}{2\sqrt{2}} 
		\lgr \plU (\phi_1 - \phi_2) ~+~ \plN ( (N-1) \phi_1 + \phi_2 ) \rgr
	~+~ g_2^2 \mu \loN ~~=~~ 0
	\\
%
\notag
&
	\p_r \poU ~+~ \frac{1}{r} \poU ~-~ \frac{1}{Nr}f \poU ~-~ \frac{N-1}{Nr} f_N \poN 
	~~+~~  \\
&\notag
\qquad\qquad\qquad\qquad\qquad\qquad
	i\, \frac{2\sqrt{2}}{N} 
		\lgr  \loU (\phi_1 + (N-1)\phi_2) ~+~ \frac{N-1}{N} \loN (\phi_1 - \phi_2) \rgr 
		~~=~~ 0
	\\
%
&
\label{N1_strans_eqn}
	\p_r \plU ~-~ \frac{1}{Nr}f \plU ~-~ \frac{N-1}{Nr} f_N \plN 
	~~+~~ \\
&\notag
\qquad\qquad\qquad\qquad\qquad\qquad
	i\, \frac{2\sqrt{2}}{N}
		\lgr \llU (\phi_1 + (N-1) \phi_2) ~+~ \frac{N-1}{N} \llN (\phi_1 - \phi_2) \rgr
		~~=~~ 0
	\\
%
&
\notag
	\p_r \poN ~+~ \frac{1}{r} \poN ~-~ \frac{1}{Nr} (f + (N-2)f_N) \poN ~-~
			\frac{1}{Nr} f_N \poU 
	~~+~~ \\
&\notag
\qquad\qquad\qquad\qquad\qquad\qquad
	i\, \frac{2\sqrt{2}}{N} 
		\lgr \loU (\phi_1 - \phi_2) ~+~ \frac{1}{N} \loN ((N-1)\phi_1 + \phi_2) \rgr
		~~=~~ 0
	\\
%
&
\notag
	\p_r \plN ~-~ \frac{1}{Nr} (f + (N-2)f_N) \plN - \frac{1}{Nr} f_N \plU 
	~~+~~  \\
&\notag
\qquad\qquad\qquad\qquad\qquad\qquad
	i\, \frac{2\sqrt{2}}{N}
		\lgr \llU (\phi_1 - \phi_2) ~+~ \frac{1}{N} \llN ((N-1)\phi_1 + \phi_2) \rgr
		~~=~~ 0
\end{align}


\pagebreak
\subsection{Superorientational zero-modes}

The ansatz is the generalization of the U(2) $ \mc{N}=1 $ zero-modes. 
In particular there is an explicit factor of two arising from the identification
\begin{align*}
&	\chi_\alpha^a ~-~ i\,\epsilon^{abc} S^b \chi_\alpha^c  ~~~\Longrightarrow~~~ 
				\chi_\alpha ~-~ [ n\ov{n},\chi_\alpha ]
		~~~\Longrightarrow~~~  2\, \xi_\alpha \ov{n} 
	\\
&	\chi_\alpha^a ~+~ i\,\epsilon^{abc} S^b \chi_\alpha^c  ~~~\Longrightarrow~~~ 
				\chi_\alpha ~+~ [ n\ov{n},\chi_\alpha ]
		~~~\Longrightarrow~~~  2\, n\ov{\xi}{}_\alpha
	~,
	\qquad\qquad  \alpha=R,L~.
\end{align*}
The zero-modes then take the form
\begin{align*}
%
	\lambda^{22\ {\rm SU(N)}} & ~~=~~ 2\, \frac{x^1 + ix^2}{r}\, \lambda_+(r) \; \xi_R\ov{n}
				~~+~~  2\, \lambda_-(r)\; n\ov{\xi}{}_R
	\\
%
	\ov{\wt{\psi}}{}_{\dot{1}} & ~~=~~ 2\, \psi_+(r)\; \xi_R \ov{n} 
				~~+~~  2\, \frac{x^1 - i x^2}{r}\, \psi_-(r)\; n\ov{\xi}{}_R~.
\end{align*}
The equations for the profiles then read
\begin{align}
%
\notag
&
	\p_r \psi_+ ~-~ \frac{1}{Nr} (f-f_N)\, \psi_+ ~+~ i\,\sqrt{2}\phi_1\,\lambda_+ ~~=~~ 0
	\\
%
\notag
	-\, & \p_r\lambda_+ ~-~ \frac{1}{r}\lambda_+ ~+~ \frac{f_N}{r}\lambda_+ 
		~+~ i\,\frac{g_2^2}{\sqrt{2}}\phi_1\, \psi_+ ~+~ \mu g_2^2\, \lambda_-  ~~=~~ 0
	\\
%
\label{N1_sorient_eqn}
&
	\p_r \psi_- ~+~ \frac{1}{r}\, \psi_- ~-~ \frac{1}{Nr}(f + (N-1)f_N)\, \psi_- 
							~+~ i\,\sqrt{2}\phi_2\, \lambda_- ~~=~~ 0
	\\
%
\notag
	-\, & \p_r\lambda_- ~-~ \frac{f_N}{r}\lambda_- ~+~ i\,\frac{g_2^2}{\sqrt{2}}\phi_2\, \psi_- 
								~+~ \mu g_2^2\, \lambda_+ ~~=~~ 0
\end{align}

\pagebreak
\subsection{Small-$\mu$ limit}

	For the supertranslational zero-modes the $ \mu^0 $-profiles are taken from Eq.~\eqref{N2_strans}:
\begin{align}
%
\notag
	\loU & ~~=~~ i\, \frac{g_1^2}{2}\, \lgr (N-1) \phi_2^2 ~+~ \phi_1^2 ~-~ N\xi \rgr  ~+~ O(\mu^2) 
	\\
%
\notag
	\loN & ~~=~~ i\, g_2^2 \lgr \phi_1^2 ~-~ \phi_2^2 \rgr ~+~ O(\mu^2)
	\\
%
\label{N1_strans_smallmu}
	\poU & ~~=~~ \frac{4\sqrt{2}}{N^2 r} \lgr \phi_1\, (f + (N-1) f_N) ~+~ (N-1)\, \phi_2\, (f-f_N) \rgr ~+~ O(\mu^2)
	\\
%
\notag
	\poN & ~~=~~ \frac{4\sqrt{2}}{N^2 r} \lgr \phi_1\, (f + (N-1) f_N) ~-~ \phi_2\, (f-f_N) \rgr ~+~ O(\mu^2)
	~.
\end{align}
	The factor of $ 1/2 $ in the first equation is due to the normalization of the U(1) gauge field and the
gaugino, Eq.~\eqref{abel_norm}.
	The $\mu^1$-solutions of Eq.~\eqref{N1_strans_eqn} in the case of equal couplings $ g_1 = g_2 $, are all proportional
to \eqref{N1_strans_smallmu} with a coefficient of $ - g^2 \mu r / 2 $:
\begin{align*}
%
	& \plU ~~=~~ -\, \frac{g^2\mu}{2}\, r \, \poU ~+~ O(\mu^3)		& \plN &~~=~~ -\, \frac{g^2\mu}{2}\, r\, \poN ~+~ O(\mu^3)  
	\\
%
	& \llU ~~=~~ -\, \frac{g^2\mu}{2}\, r \, \loU ~+~ O(\mu^3)		& \llN &~~=~~ -\, \frac{g^2\mu}{2}\, r\, \loN ~+~ O(\mu^3)~.
\end{align*}

For the superorientational zero-modes the zero-order in $ \mu $ profiles are taken from Eq.~\eqref{N2_sorient},
\begin{align}
%
\notag
 	\lambda_+(r) & ~~=~~ -\, \frac{i}{\sqrt{2}} \frac{f_N}{r} \frac{\phi_1}{\phi_2}  ~+~ O(\mu^2) \\
%
\label{N1_sorient_smallmu_plus}
	\psi_+(r) & ~~=~~ -\, \frac{\phi_1^2 ~-~ \phi_2^2}{2\phi_2} ~+~ O(\mu^2),
\end{align}
and the overall signs are correlated with the right-handed zero-modes in \eqref{N2_sorient}.
The leading-order contributions to the $ \psi_- $ and $ \lambda_- $ profile functions are straightforwardly taken from the $ U(2) $ theory:
\begin{align*}
%
	\psi_- & ~~=~~ -\, \mu g_2^2 \frac{r}{4\phi_1} \left( \phi_1^2 ~-~ \phi_2^2 \right)  ~+~ O(\mu^3)
	\\ 
%
	\lambda_- & ~~=~~ -\, \mu g_2^2 \frac{i}{2\sqrt{2}} \lgr (f_N - 1) \frac{\phi_2}{\phi_1} ~+~ \frac{\phi_1}{\phi_2} \rgr ~+~ O(\mu^3)~.
\end{align*}
Again, the overall signs are correlated with \eqref{N1_sorient_smallmu_plus} and \eqref{N2_sorient}.


%
%\begin{align*}
%%
%\mc{L}_{superorient} ~~=~~ &
%\int dx^0 dx^3 \frac{1}{2\beta} 
%	\left\lgroup 
%	\ov{\xi}_L \, i (\md_0 ~+~ i \md_3 ) \xi_L ~~+~~ \ov{\xi}_R \, i (\md_0 ~-~ i \md_3 ) \xi_R 
%	\right\rgroup 
%	\times 
%	\\
%%
%& \qquad\qquad \int dx^1 dx^2 
%	\left\{  
%		\frac { (\phi_1^2 ~-~ \phi_2^2)^2 } { \phi_2^2 }
%		~~+~~
%		\frac {4}{r^2 g_2^2}
%		\left( \frac{\phi_1}{\phi_2}\,
%			f_N 
%		\right)^2 
%	\right\}  & 
%	\\
%%
%\mc{L}_{supertrans} ~~=~~ &
%	\int dx^0 dx^3 \frac{1}{2\beta} 
%	\lgr
%		\ov{\zeta}_R \, i (\p_0 - i \p_3 ) \zeta_R ~+~
%		\ov{\zeta}_L \, i (\p_0 + i \p_3 ) \zeta_L 
%	\rgr
%	\times
%	\\
%%
%& \qquad\qquad \int dx^1 dx^2 
%	\Bigl\{  
%		8\, (\p_r \phi_1)^2   ~~+~~ 8\, (N-1) (\p_r \phi_2)^2 
%		~~+~~
%%	\right.
%	\\
%%
%&\qquad\qquad
%%	\left.
%	\qquad\qquad
%		g_1^2 \left( (N-1) \phi_2^2 ~+~ \phi_1^2 ~-~ N\xi \right)^2
%		~~+~~
%		2 g_2^2 \frac{N-1}{N} (\phi_1^2 ~-~ \phi_2^2 )^2 
%	\Bigr\}
%	\\
%\end{align*}
%
\vspace{5.0cm}




\end{document}
